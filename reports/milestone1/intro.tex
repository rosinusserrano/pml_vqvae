Over the course of this project, we aim to implement a \ac{vq} as introduced by~\cite{vqvae}
and to reproduce the results of the authors.
The machine learning model described in the paper learns
\begin{itemize}
    \item \textbf{Compression}: Encoding real world data into learned discrete representations and decoding
    these back to the original data space with minimal error
    \item \textbf{Generation}: Generating new, realistic samples in the representation space, which can also be
    decoded to create realistic samples in the original data space
\end{itemize}
To begin, we focus solely on the field of 2D images, specifically using the ImageNet and CIFAR-10 datasets.
In the later stages of this project, we will either try to further increase the models capabilities in this one field,
or widen our work over the other modalities touched upon by the paper introducing the model, i.e.\ audio and video
frames.

For the first milestone, we explored the datasets and implemented a baseline method for subsequent comparison to our
final model.

This report is structured as follows: We begin with an overview of the aforementioned datasets and the preprocessing and
data-cleaning steps we apply.
The next section describes the evaluation metrics suitable to judge the performance of our model, followed by a
section on the baseline method(s) we implement and use.
We close with a discussion of the results of said baseline models, questions we would like to solve over the course of
the project and possible real world use-cases of the final model.
Since an in-depth explanation of the \ac{vq} model is beyond the scope of this report, we refer the reader to the
original paper,
which we also mean to do whenever we use the term ``the paper'' without any additional citation.