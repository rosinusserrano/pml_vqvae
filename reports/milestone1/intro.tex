Over the course of this machine learning project, we aim to implement a \ac{vq} as introduced by~\cite{vqvae}
and to reproduce the results of the authors.
To begin, we focus solely on the field of 2D images, specifically using the ImageNet and CIFAR-10 datasets.
In the later stages of this project, we will either try to further increase the models capabilities in this one field,
or widen our work over the other modalities touched upon by the paper introducing the model (e.g.\ audio, video frames).

For the first milestone, we explored the datasets and implemented a baseline method for subsequent comparison to our
final model.

This report is structured as follows: We begin with an overview of the aforementioned datasets and the preprocessing and
data-cleaning steps we apply.
The next section describes the baseline method we implement, followed by a section on the evaluation metrics suitable
for both these baselines and our future \ac{vq} model.
We close with a discussion of the results of our baseline models, questions we would like to solve over the course of
the project and possible real world use-cases of the final model.
Since an explanation of the \ac{vq} model is beyond the scope of this report, we refer the reader to the original paper.
