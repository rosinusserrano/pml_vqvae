\documentclass[10pt,a4paper,twoside]{article}
\usepackage[a4paper,top=20mm,bottom=20mm,outer=5cm]{geometry}
\usepackage[utf8]{inputenc}
\usepackage[english]{babel}
\usepackage{graphicx}
\usepackage{hyperref}
\usepackage{amsmath}
\usepackage{acronym}
\usepackage{cleveref}
\usepackage{natbib}
\bibliographystyle{abbrvnat}
\setcitestyle{authoryear}


\title{Project Machine Learning\\--- Milestone 1 ---} 

%%%%%%%%%%%%%%%%%%%%%%%%%%%%%%%%%%%%%%%
%                                     %
%   EVERYTHING BELOW CAN BE CHANGED   %
%                                     %
%%%%%%%%%%%%%%%%%%%%%%%%%%%%%%%%%%%%%%%

\author{Konstantin Ausborn, Timon Palm, Marco Rosinus Serrano}
\date{\today}

\begin{document}
\acrodef{AE}{Autoencoder}
\acrodef{VAE}{Variational Autoencoder}
\acrodef{VQ}[VQ-VAE]{Vector Quantized Variatonal Autoencoder}

\maketitle

\section{Dataset overview}
    \begin{align}
        1 + 1 = 3
    \end{align}

\section{Baseline method and evaluation}\label{sec:baseline-method-and-evaluation}
Since it is the original Model that both \ac{VQ} and now the \ac{VAE} have been developed from,
we chose a basic \ac{AE} as our baseline method.

\begin{itemize}
    \item basically just use negative log entropy as the factor to compare
    \item this describes the entropy of the picture generated?
    I do not have clarity here see: https://bjlkeng.io/posts/a-note-on-using-log-likelihood-for-generative-models/
    \item I need to understand PixelCNN better to continue
    \item see Shannon for theory on entropy\cite{shannon}
    \item dont use parzen windows\cite{note_on_eval}
    \item
    \item
\end{itemize}

\section{Discussion}
:-)

\bibliography{bibliography}
\end{document}
