\subsection{Baseline Results}\label{subsec:baseline-results}
\subsubsection{Class Differences}\label{subsubsec:class-differences}
\subsection{Further Ideas for the Project}\label{subsec:further-ideas}
\subsection{Real World Applications}\label{subsec:real-world-applications}

\vskip 1em
\textbf{A Note on Overfitting:}
Overfitting refers to the phenomenon where a model starts to memorize the training data instead of learning useful
features, in turn not generalizing well to unseen data.

While theoretically, this may very well be an issue with \ac{ae} and \ac{vae}, we have not experienced this happening
yet with either when training with adequate training-set sizes.
This is likely due to the large dataset sizes we use and the inductive bias of the convolution operation~
\cite{citationNeeded}.